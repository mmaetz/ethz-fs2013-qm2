``solve'' numerically with iterative procedure. start with ``guess'' for $\varphi_{i}^{(0)}$.
$\leadsto$ into Eq. \ref{eq:hartree2} $\leadsto V_{i}^{(0)}$
$\leadsto$ into Eq. \ref{eq:hartree1} solve $\leadsto \varphi_{i}^{(1)}$ $\leadsto$ etc.

physical interpretation of Lagrange multipliers $\varepsilon_i$
$\varphi_{i}^{*}\cdot \ref{eq:hartree1}$
\begin{dgroup}[]
	\begin{dmath}[]
		\hiderel{\Rightarrow} \int_{}^{}\rd^3r\,
		\left( \frac{-\hbar^2}{2m}\abs{\nabla_i\varphi_i}^2+\left( -\frac{Ze^2}{r_i}+V_i \right)\abs{\varphi_i}^2 \right)=\varepsilon_i
	\end{dmath}
	\begin{dsuspend}
		with $\varepsilon_i$ the ionization energy of $i$-th electron, assuming others are not affected.
	\end{dsuspend}
\end{dgroup}
\section{Hartree-Fock approximation}
improved ansatz for 
\begin{dgroup}[]
	\begin{dmath}[]
		\psi\left( 1,\ldots,N \right)=\frac{1}{\sqrt{N!}}
		\begin{vmatrix}
			\varphi_1(1)&\cdots&\varphi_N(1)\\
			\vdots&\ddots&\vdots\\
			\varphi_1(N)&\cdots&\varphi_N(N)
		\end{vmatrix}
	\end{dmath}
\end{dgroup}
fully comptible with Pauli $\leadsto$ as for Hartree, plug into $H$ and minimize. Eq. \ref{eq:hartree1} stayse the same. Eq. \ref{eq:hartree2} 
\begin{dgroup}[]
	\begin{dmath}[]
		\frac{1}{2}\sum_{j\neq i}^{}\int_{}^{}\rd^3 \vtr{r}_i\int_{}^{}\rd^3 \vtr{r}_j\frac{e^2}{\abs{\vtr{r}_i-\vtr{r}_j}}
		\times \left( \varphi_{i}^{*}(r_i)\varphi_{j}^{*}(r_j)\varphi_i(r_i)\varphi_j(r_j)-\varphi_{i}^{*}(r_i)\varphi_{j}^{*}(r_j)\varphi_i(r_j)\varphi_j(r_i) \right)
	\end{dmath}
	\begin{dsuspend}
		with $\varphi_{i}^{*}(r_i)\varphi_{j}^{*}(r_j)\varphi_i(r_i)\varphi_j(r_j)$ the Hartree term and $\varphi_{i}^{*}(r_i)\varphi_{j}^{*}(r_j)\varphi_i(r_j)$ the exchange term.
	\end{dsuspend}
\end{dgroup}
To understand exchange term consider $N=2$
\begin{dgroup}[]
	\begin{dmath}[]
		\psi\left( 1,2 \right)=\frac{1}{\sqrt{2!}}\left( \varphi_i(1)\varphi_2(2)-\varphi_1(2)\varphi_2(1) \right)
	\end{dmath}
	\begin{dsuspend}
		``new'' in H-F write down all terms for
	\end{dsuspend}
	\begin{dmath*}[]
		\mel{\psi}{\frac{1}{\abs{\vtr{r}_1-\vtr{r}_2}}}{\psi}
		=\frac{1}{2}\int_{}^{}\rd^3 \vtr{r}_1\int_{}^{}\rd^3 \vtr{r}_2
		\left( \varphi_{1}^{*}(r_1)\varphi_{2}^{*}(r_2)\varphi_1(r_1)\varphi_2(r_2)+\text{``$1\hiderel{\leftrightarrow}2$''} 
		-\varphi_{1}^{*}(r_1)\varphi_{2}^{*}(r_2)\varphi_1(r_2)\varphi_2(r_1)-\text{``$1\hiderel{\leftrightarrow} 2$''} 
		\right)
	\end{dmath*}
\end{dgroup}
\section{The periodic table and Hund's rules}
Electron in atom feels effective potential $V_{\text{eff}}$ (from nucleus and other electron) which is spherically symmetric.
\begin{dgroup}[]
	\begin{dmath}[]
		\psi_i=R_{n\ell}(r)Y_{\ell}^{m}\left( \theta,\varphi \right)\chi_{m_{s}}
	\end{dmath},
	\begin{dsuspend}
		with $\chi_{m_s}$ the spin and $R_{n\ell}$ different from hydrogen
	\end{dsuspend}
\end{dgroup}
\paragraph{general rules}
\begin{dmath}[]
	\begin{cases}
		n\text{ small}&\text{stronger binding}\\
		\ell\text{ small}&\text{electron is closer to nucleus}
	\end{cases}
\end{dmath}
``compete sometimes''. for each $n:$
\begin{dgroup}[]
	\begin{dmath}[]
		\begin{array}[]{lcccc}
			\ell&0&1&2&3\\
			\text{name}&S&P&D&F\\
			\deg 2\left( 2\ell +1 \right)&2&4&6&10
		\end{array}
	\end{dmath}
	\begin{dmath}[]
		\begin{array}[]{lccccc}
			K\text{-shell}&n=1&\ell=0&2\text{ elements}&H,He&(1s)\\
			L\text{-shell}&n=2&\ell=0&2&Li,Be&(2s)\\
			&&\ell=1&6&B-Ne&(2s)\\
			M\text{-shell}&n=3&\ell=0&2&Na-Mg&(3s)\\
			&&\ell=1&6&Al-Ar&(3s)\\
			&&\ell=2&\ldots&\ldots&(3p)\\
			N\text{-shell}&n=4&\ell=0&2&&(4s)\\
			&n=3&\ell=2&10&Al-Ar&(3d)\\
			&n=4&\ell=1&6&\ldots&(4p)\\
		\end{array}
	\end{dmath}
\end{dgroup}
