configuration of electron $\to$ chemical properties of elements. What is the configuration (total spin, $L$, $J$) ofthe outer electron. $\to$ Hund's rules (empirical) ~ [Notation ${^{2s+1}L_{J}}$]
\begin{expl}
	Carbon $\left( 1s \right)^2\left( 2s \right)^2(2p)^2$\\
	for each of the $2$ $2p$-electrons $2p$-electrons we can have $m_{\ell}=-1,0,1$, $m_{s}=-\frac{1}{2},\frac{1}{2}$. $\to 6$ possibilities\\
	for both 
	\begin{dmath}[]
		\frac{6\cdot 5}{2}=15\text{ possibilities}
	\end{dmath}
\end{expl}
\begin{dgroup}[]
	\begin{dmath}[]
		L=
		\begin{matrix}
			0,2\\
			1
		\end{matrix}
		\begin{matrix}
			\text{symmetric}\\
			\text{anti-symmetric}
		\end{matrix}
	\end{dmath}
	\begin{dmath}[]
		S=
		\begin{matrix}
			1\\
			0
		\end{matrix}
		\begin{matrix}
			\text{symmetric}\\
			\text{anti-symmetric}
		\end{matrix}
	\end{dmath}
\end{dgroup}
total wave function is antisymmetric
\begin{dmath}[]
	\begin{array}[]{ccccc}
		L&S&J&{}^{2s+1}L_{J}&\text{deg}\\
		\hline
		0&0&0&{}^1S_0&1\\
		&&0&{}^3P_0&1\\
		1&1&1&{}^3P_1&3\\
		&&2&{}^3P_2&5\\
		2&0&2&{}^1D_2&5\\
		&&&&15\\
	\end{array}
\end{dmath}
Which one is ground state? $\to$ Hund's rules
\begin{enumerate}[(1)]
	\item make spin maximal
		\begin{dmath}[]
			\to \text{spin part more symmetric}
			\to \text{orbital port more asymmetric}
			\to \text{electron further away from each other}
			\to \text{less repulsion}
		\end{dmath}
		for $C$: $s=1$
	\item make $L$ maximal
		\begin{dmath}[]
			\to \text{electron average further away from each other}
			\to \text{less repulsion}
		\end{dmath}
		no impaft for $C$
	\item 
		\begin{dgroup}[]
			\begin{dmath}[]
				\Delta E_{\text{SO}}=\text{constant} \left(  j\left( j+1 \right)-\ell\left( \ell+1 \right)-s\left( s+1 \right) \right)
			\end{dmath}
			\begin{dmath}[]
				\text{constant}
				\begin{cases}
					>0&\text{if subshell \emph{no} more than half filled }J=\abs{L-S}\\
					<0&\text{if subshell more than half filled }J=\abs{L+S}
				\end{cases}
			\end{dmath}
		\end{dgroup}
		for $C$ first case and 
		\begin{dmath}[]
			J=\abs{L-S}=0
		\end{dmath}
		ground state ${}^3P_0$
\end{enumerate}
\chapter{Approximation methods for time-dependent problems}
We now want to know time evolution
\begin{dmath}[]
	\ket{\psi(t)}=U\left( t,t_0 \right)\ket{\psi(t_0)}
\end{dmath}
We know:
\begin{dmath}[]
		i\hbar\pdv{}{t}\ket{\psi(t)}=H(t)\ket{\psi(t)}
\end{dmath}
with $H(t)$ now time dependent $\to$ more complicated relation between $H$ and $U$
\section{Time-dependent perturbation theory}
Let
\begin{dgroup}[]
	\begin{dmath}[]
		H(t)=H_0+\lambda V(t)
	\end{dmath},
	\begin{dsuspend}
		with $H_0$ time independent and that can be solved and $V(t)$ with $t$ the ``only'' difference to chapter 1.
	\end{dsuspend}
	\begin{dmath}[]
		H_0\ket{\psi_{n}^{(0)}}=E_{n}^{(0)}\ket{\psi_{n}^{(0)}}
	\end{dmath}
	\begin{dmath}[]
		i\hbar \pdv{}{t}\ket{\psi(t)}=H(t)\ket{\psi(t)}
		=\left( H_0+\lambda V(t) \right)\ket{\psi(t)}
	\end{dmath}
	\begin{dsuspend}
		for any $t$:
	\end{dsuspend}
	\begin{dmath}[]
		\ket{\psi(t)}=\sum_{n}^{}c_n(t)e^{-\frac{i}{\hbar}E_{n}^{(0)}}\ket{\psi_{n}^{(0)}}
	\end{dmath}
	\begin{dmath}[]
		\braket{\psi(t)}{\psi(t)}=1
	\end{dmath}
	\begin{dmath}[]
		\hiderel{\Rightarrow}\sum_{n}^{}\abs{c_n}^2=1
	\end{dmath}
	\begin{dsuspend}
		we can also write
	\end{dsuspend}
	\begin{dmath}[]
		V(t)\ket{\psi_{n}^{(0)}}=\sum_{m}^{}\ket{\psi_{m}^{(0)}}\mel{\psi_{m}^{(0)}}{V(t)}{\psi_{n}^{(0)}}
	\end{dmath}
	\begin{dsuspend}
		with 
	\end{dsuspend}
	\begin{dmath}[]
		\mel{\psi_{m}^{(0)}}{V(t)}{\psi_{n}^{(0)}} =V_{mn}(t)
	\end{dmath}
	\begin{dsuspend}
		$\to$ into Schrödinger
	\end{dsuspend}
	\begin{dmath}[]
		\sum_{n}^{}\left( i\hbar \dot{c}_n+E_{n}^{(0)}c_n \right)\ket{\psi_{n}^{(0)}}
		=\sum_{n}^{}c_ne^{-\frac{i}{\hbar}E_{n}^{(0)}t}\left( E_{n}^{(0)}\ket{\psi_{n}^{(0)}}+\lambda\sum_{m}^{}V_{mn}(t)\ket{\psi_{m}^{(0)}} \right)
	\end{dmath}
	\begin{dsuspend}
		swap labels $m$ and $n$ on rhs
	\end{dsuspend}
	\begin{dmath}[]
		\sum_{n}^{}i\hbar\dot{c_n}e^{-\frac{i}{\hbar}E_{n}^{(0)}t}\ket{\psi_{n}^{(0)}}
		=\sum_{n,m}^{}\lambda c_me^{-\frac{i}{\hbar}E_{m}^{(0)}t}V_{nm}(t)\ket{\psi_{n}^{(0)}}
	\end{dmath}
	\begin{dmath}[]
		\hiderel{\Rightarrow}
		\dot{c_n}=\left( i\hbar \right)^{-1}\lambda \sum_{m}^{}V_{mn}e^{\frac{i}{\hbar}\left( E_{n}^{(0)}-E_{m}^{(0)} \right)t}c_m
	\end{dmath}
	\begin{dsuspend}
		with
	\end{dsuspend}
	\begin{dmath}[]
		\omega_{nm}=\frac{E_{n}^{(0)}-E_{m}^{(0)}}{\hbar}
	\end{dmath}
	\begin{dsuspend}
		so
	\end{dsuspend}
	\begin{dmath}[]
		\dot{c_n}=\left( i\hbar \right)^{-1}\lambda\sum_{m}^{}V_{nm}e^{i\omega_{nm}t}c_m
	\end{dmath}.
	\begin{dsuspend}
		Now expand in $\lambda$ ($\to$ perturbation theory)
	\end{dsuspend}
	\begin{dmath}[]
		c_n=c_{n}^{(0)}+\lambda c_{n}^{(1)}+\lambda^2 c_{n}^{(2)}+\ldots
	\end{dmath}
	\begin{dmath}[]
		\dot{c}_{n}^{(0)}=0\condition*{\mathcal{O}(\lambda^0)}
	\end{dmath}
	\begin{dmath}[]
		c_{n}^{(1)}=\left( i\hbar \right)^{-1}\sum_{m}^{}V_{nm}e^{i\omega_{nm}t}c_{m}^{(0)}
	\end{dmath}
	\begin{dsuspend}
		\ldots
	\end{dsuspend}
	\begin{dmath}[]
		c_{n}^{(j)}=\left( i\hbar \right)^{-1}\sum_{m}^{}V_{nm}e^{i\omega_{nm}t}c_{m}^{\left( j-1 \right)}
	\end{dmath}
	\begin{dsuspend}
		Let system be in state $\ket{\psi_{i}^{(0)}}$ at time to initial condition
	\end{dsuspend}
	\begin{dmath}[]
		c_{m}^{(0)}=\delta_{im}
	\end{dmath}
	\begin{dmath}[]
		\dot{c}_{f}^{(1)}=\left( i\hbar \right)^{-1}V_{fi}e^{i\omega_{fi}t}
	\end{dmath}
	\begin{dmath}[]
		c_{f}^{(1)}(t)=\left( i\hbar \right)^{-1}\int_{t_0}^{t}\dd{t'}V_{fi}(t')e^{i\omega_{fi}t'}
	\end{dmath}
	\begin{dsuspend}
		$\to$ transition probability for the system to be found in state $\ket{\psi_{f}^{(0)}}$ at time $t$.
	\end{dsuspend}
	\begin{dmath}[]
		P_{i\to f}=\abs{c_{f}^{(1)}}^2
		=\frac{1}{\hbar^2}\abs{\int_{t_0}^{t}\dd{t'}\;V_{fi}e^{i\omega_{fi}t'}}^2+\mathcal{O}\left( \lambda^2 \right)
	\end{dmath}
	\begin{dsuspend}
		approximation only valif if
	\end{dsuspend}
	\begin{dmath}[]
		\abs{c_f}^2\ll1
	\end{dmath}
	\begin{dsuspend}
		Higher orders in $\lambda$ will be covered  section 4.4 and Exercise
	\end{dsuspend}
\end{dgroup}
\section{Constant perturbation}
Let
\begin{dgroup}[]
	\begin{dmath}[]
		V(t)=
		\begin{cases}
			0&\text{for }t<t_0(=0)\\
			V&\text{(constant) for }t>t_0
		\end{cases}
	\end{dmath}
	\begin{dmath}[]
		P_{i\to f}=\frac{1}{\hbar^2}\abs{V_{fi}}^2\abs{\int_{t_0=0}^{t}\dd{t'}\, e^{i\omega_{fi}t'}}
	\end{dmath},
	\begin{dsuspend}
		using
	\end{dsuspend}
	\begin{dmath}[]
		\int_{t_0=0}^{t}\dd{t'}\, e^{i\omega_{fi}t'}
		=\frac{2}{\omega^2}\left( 1-\cos\omega_{fi}t \right)
		=\frac{4}{\omega^2}\sin^2\left( \frac{\omega_{fi}t}{2} \right)
	\end{dmath},
	\begin{dsuspend}
		and
	\end{dsuspend}
	\begin{dmath}[]
		\delta_{t}(\alpha)\equiv \frac{\sin^2\left( \alpha t \right)}{\pi\alpha^2 t}
		=
		\begin{cases}
			\frac{t}{\pi}&\alpha=0\\
			<\frac{1}{\pi\alpha^2 t}&\alpha\neq=0
		\end{cases}
	\end{dmath},
	\begin{dsuspend}
		which is plotted in Fig. \ref{fig:sinc1}
	\end{dsuspend}
	\begin{dmath}[]
		\lim_{t\to\infty}P_{i\to f}=\frac{\pi t}{\hbar^2}\abs{V_{fi}}^2\delta\left( \frac{E_{f}^{(0)}-E_{j}^{(0)}}{2\hbar} \right)
	\end{dmath}
	\begin{dmath}[]
		P_{i\to f}=\frac{2\pi t}{\hbar}\abs{V_{fi}}^2\delta\left( E_{f}^{(0)}-E_{i}^{(0)} \right)
	\end{dmath}
	\begin{dsuspend}
		transition rate$= \text{probability}/\text{time}$
	\end{dsuspend}
	\begin{dmath}[]
		\Gamma_{fi}=\frac{2\pi}{\hbar}\abs{V_{fi}}^2\delta\left( E_f-E_i \right)
	\end{dmath}
\end{dgroup}
\begin{figure}[]
	\begin{center}
		\begin{tikzpicture}[]
			\coordinate[label=above:$\delta(\alpha)$] (y) at (0,4);
			\coordinate[label=right:$\alpha t$] (x) at (4,0);
			\draw[-latex] (0,0) -- (y);
			\draw[-latex] (-4,0) -- (x);
			\draw[domain=-4:4] plot[id=sinc,smasa] function{sin(x*pi)**2/x**2};
		\end{tikzpicture}
	\end{center}
	\caption{}
	\label{fig:sinc1}
\end{figure}
Consider transitions into continuous specturm $\rho(E)$
\begin{dgroup}[]
	\begin{dmath}[]
		\int_{E_1}^{E_2}\dd{E}\,\rho(E)=\text{number of states in energy range $E_1$-$E_2$}
	\end{dmath}
	\begin{dmath}[]
		\sum_{f}^{}\Gamma_{fi}\hiderel{\to}\int_{}^{}\dd{E}\,\rho(E)\Gamma_{fi}
		=\frac{2\pi}{\hbar}\rho(E_f)\abs{V_{fi}}^2
	\end{dmath}
	\begin{dsuspend}
		golden rule!
	\end{dsuspend}
\end{dgroup}
requires contiuum of states and applicabiliity of perturbation theory.
