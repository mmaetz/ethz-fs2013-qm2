\section{Dipole approximation and selection rules}
Transitions are governed by $\alpha^{*}\mel{\psi_0}{j\left( \vtr{k} \right)\cdot \vtr{\upepsilon}^{*}\left( \vtr{k} \right) }{\psi_n}$
\begin{dgroup}[]
	\begin{dmath}[]
	j\left( \vtr{k} \right)=\frac{1}{2m}\left( \vtr{p} e^{-i\vtr{k}\cdot \vtr{r}}]+e^{-i\vtr{k}\cdot \vtr{r}}\vtr{p} \right)
	=\frac{\vtr{p}}{m}
	\end{dmath}
	\begin{dsuspend}
		with Dipole approximation
	\end{dsuspend}
	\begin{dmath}[]
		e^{-i\vtr{k}\vtr{r}}=1-i\vtr{k}\vtr{r}+\ldots
	\end{dmath}
	\begin{dmath}[]
		k\sim \frac{1}{\lambda}\condition{for visible light $\sim \left( \SI{10e-7}{\meter} \right)^{-1}$}
	\end{dmath}
	\begin{dmath}[]
		r\sim \text{size of atom}\sim \text{Bohr radius}\sim \SI{10e-10}{\meter}
	\end{dmath}
	\begin{dmath}[]
		\vtr{j}\cdot \vtr{A}\to \text{Dipole approx.}: \vtr{j}\hiderel{=}\frac{\vtr{p}}{m}
	\end{dmath}
	\begin{dsuspend}
		(radiation filed $\sim$ constant within atom)
	\end{dsuspend}
	\begin{dsuspend}
		Aside: this is equivalent to a term
	\end{dsuspend}
	\begin{dmath}[]
		e\vtr{r}\cdot \vtr{E}=-\vtr{d}\cdot \vtr{E}
	\end{dmath}
	\begin{dsuspend}
		with $\vtr{d}$ dipole moment
	\end{dsuspend}
	\begin{dmath}[]
		\vtr{d}\equiv -e \vtr{r}
	\end{dmath}
	\begin{dsuspend}
		(compare $\vtr{\upmu}\cdot \vtr{B}$)
	\end{dsuspend}
\end{dgroup}
\begin{proof}
	\begin{dgroup}[]
		\begin{dmath}[]
			\comm{r_x}{H_0}=\comm{r_x}{\frac{\vtr{p}^2}{2m}}
			=\frac{1}{2m}\left( \comm{r_x}{\vtr{p}}\vtr{p}+\vtr{p}\comm{r_x}{p} \right)
		\end{dmath}
		\begin{dsuspend}
			with
		\end{dsuspend}
		\begin{dmath}[]
			\comm{r_x}{\vtr{p}}=\sum_{y}^{}i\hbar \delta_{xy}p_y
		\end{dmath}
		\begin{dmath}[]
			\comm{r_x}{H_0}=\frac{p_x}{m}i\hbar
		\end{dmath}
		\begin{dmath}[label=eq:gruenerstern]
			\comm{r}{H_0}=i\hbar\frac{\vtr{p}}{m}
		\end{dmath}
		\begin{dsuspend}
			emission
		\end{dsuspend}
		\begin{dmath}[]
			\mel{\psi_0}{e\vtr{r}\cdot \vtr{E}}{\psi_n}
			=-\frac{e}{c}\mel{\psi_0}{\vtr{r}\cdot \dot{\vtr{A}}}{\psi_n}
			=-\frac{e}{c}i\omega \mel{\psi_0}{\vtr{r}\cdot \vtr{A}}{\psi_n}
			=-\frac{e}{c}\frac{i}{\hbar}\left( E_n-E_0 \right)\mel{\psi_0}{\vtr{r}\cdot \vtr{A}}{\psi_n}
			=-\frac{e}{c}\frac{i}{\hbar}\mel{\psi_0}{\comm{\vtr{r}}{H_0}\cdot \vtr{A}}{\psi_n}
		\end{dmath}
		\begin{dsuspend}
			using Eq. \ref{eq:gruenerstern}
		\end{dsuspend}
		\begin{dmath}[]
			\mel{\psi_0}{e\vtr{r}\cdot \vtr{E}}{\psi_n}
			=-\frac{e}{c}\mel{\psi_0}{\frac{\vtr{p}}{m}\cdot \vtr{A}}{\psi_n}
			=\frac{e}{c}\mel{\psi_0}{\vtr{j}\cdot \vtr{A}}{\psi_n}
		\end{dmath}
	\end{dgroup}
\end{proof}
\begin{proof}[alternative]
	Make gauge trafo with 
	\begin{dgroup}[]
		\begin{dmath}[]
			\chi\left( \vtr{r},t \right)=-\vtr{A}(t)\cdot \vtr{r}
		\end{dmath}
		\begin{dmath}[compact]
			\vtr{A}\to \vtr{A}+\vtr{\nabla}\xi =0
		\end{dmath}
		\begin{dmath}[compact]
			0=\Phi\to \Phi -\frac{1}{c}\dot{\chi}
			=-\frac{1}{c}\dot{\vtr{A}}\cdot \vtr{r}=-\vtr{E}\cdot \vtr{r}
		\end{dmath}
		\begin{dmath}[]
			H-e\Phi=-e\vtr{r}\vtr{E}
		\end{dmath}
	\end{dgroup}
\end{proof}
Atomic transitions are only possible if
\begin{dgroup}[]
	\begin{dmath}[]
		\mel{\psi_{n'\ell' m_{\ell}'}}{\vtr{r}}{\psi_{n\ell m_{\ell}}}\neq 0
	\end{dmath}
\end{dgroup}
Given $\ell, m_{\ell}$ this imposes constraints on $\ell',m_{\ell'}\to$ selection rules (in dipole approximation). Can be obtained by (solving at properties of spherical harmonics).
\begin{dgroup}[]
	\begin{dmath}[]
		\int_{}^{}\dd{\Omega}\, \left( Y_{\ell'}^{m_{\ell'}}\left( \theta,\phi \right) \right)^{*}
		\begin{pmatrix}
			x+iy\\
			x-iy\\
			z
		\end{pmatrix}
		\mathcal{Y}_{\ell}^{m_{\ell}}\left( \theta,\phi \right)
	\end{dmath}
	\begin{dsuspend}
		this is most of the time $0$ exept if
	\end{dsuspend}
	\begin{dmath}[]
		\mel{\psi_{n'\ell'm_{\ell}'}}{z}{\psi_{n\ell m_{\ell}}}\neq 0
		\condition*{\ell'=\ell\pm 1,m_{\ell}'=m_{\ell}}
	\end{dmath}
	\begin{dmath}[]
		\mel{\psi_{n'\ell'm_{\ell}'}}{x\pm iy}{\psi_{n\ell m_{\ell}}}\neq 0
		\condition*{\ell'=\ell \pm 1, m_{\ell}'=m_{\ell}\pm 1}
	\end{dmath}
	\begin{dsuspend}
		(or use Wigner-Ekcart theorem, $\vtr{r}$ is a vector operator)\\
		$\Rightarrow$ selection rules for $E_{1}$ tronsitions (dipole approximation)
	\end{dsuspend}
	\begin{dmath}[]
		\Delta \ell =\pm 1
	\end{dmath}
	\begin{dmath}[]
		\Delta m =0, \pm 1
	\end{dmath}
\end{dgroup}
These rules are violated by ``beyound-dipole'' transitions (e.g. $E_{2}$ quadrupole transitions).

Further selection rules: $\Delta S=0$ (spin part of wavefunction not affected by $\vtr{r}\cdot \vtr{E}$) always true: \emph{no} transitions between $j=0\to j=0$ (total angular momentum conservation)
