%% Theorem-like environments

%% This can be changed according to language. You can comment out the ones you
%% don't need.

%% German theorems
%\newtheorem{bspl}{Beispiel}[section]
%\newtheorem{bemk}{Bemerkung}[section]
%\newtheorem{korl}{Korrolar}[section]
%\newtheorem{satz}{Satz}[section]

%% English variants
\newtheorem{expl}{Example}[section]
\newtheorem{remk}{Remark}[section]
\newtheorem{corl}{Corollary}[section]
\newtheorem{thrm}{Theorem}[section]

%% Theorems common to both languages
\newtheorem{defn}{Definition}[section]
\newtheorem{lemm}{Lemma}[section]
\newtheorem{prop}{Proposition}[section]

%% Note, if you want to run all of them on a single counter, use
%\newtheorem{expl}{Example}[section]
%\newtheorem{remk}[expl]{Remark}
%\newtheorem{corl}[expl]{Corollary}
%% etc. You can also run equations on the same counter if you like,
%% with
%\numberwithin{equation}{section}
%\newtheorem{expl}[equation]{Example}
%% You can also declare unnumbered environments with
%\newtheorem*{expl}{Example}

%% Proof environment with a small square as a "qed" symbol
\theoremstyle{nonumberplain}
%\theoremsymbol{\ensuremath{\square}}
\newtheorem{proof}{Proof}
%\neotheorem*{beweis}{Beweis}
