%\definecolor{scolor}{RGB}{255,220,181} 
%\definecolor{scolor}{RGB}{255,166,0} 
%\colorlet{scolordark}{scolor!50!black}

\definecolor{rioday}{RGB}{255,166,0} 
\colorlet{riomorning}{rioday!65!black}
\colorlet{rionight}{rioday!50!black}
%\definecolor{scolor2}{RGB}{0,255,166} 
%\definecolor{scolor2}{RGB}{100,100,170} 
%\definecolor{scolor2}{RGB}{0,233,255} 
%\definecolor{scolor2}{RGB}{0,182,255} 
%\definecolor{scolor2}{RGB}{0,255,16} 
%\definecolor{scolor2}{RGB}{0,89,255} 
%\colorlet{scolor2dark}{scolor2!50!black}

\definecolor{helsinkiday}{RGB}{0,89,255} 
%%% Hi from 2013. I think it's fascinating that this code could possibly be used until like 2043 or even further.
\colorlet{helsinkimorning}{helsinkiday!65!black}
\colorlet{helsinkinight}{helsinkiday!50!black}
\hypersetup{colorlinks=false,linkbordercolor=rioday}
%%%This is the global setting of the number of samples used to plot the functions in the graphs. 
\tikzset{
	smasa/.style={
		%%%The number of samples with low number.
		%%% 1000 should be enough for a final edition
		samples=10
		%samples=1000
	},
	bigsa/.style={
		%%%The number of samples with high number.
		%%% 10000 should be enough for a final edition.
		samples=20
		%samples=10000
	},
	%%%The standard filling
	sfill/.style={
		fill=rioday
	},
	zyplane/.style={canvas is zy plane at x=#1,very thin},
	zxplane/.style={canvas is zx plane at y=#1,very thin},
	yxplane/.style={canvas is yx plane at z=#1,very thin}
}
